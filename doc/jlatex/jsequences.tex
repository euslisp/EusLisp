\section{列、行列とテーブル}
\markright{\arabic{section}. 列、行列とテーブル}

\subsection{一般列}

ベクトル(1次元行列)とリストは、一般の列である。
文字列(string)は、文字(character)のベクトルなので、列である。

{\bf map, concatenate}や{\bf coerce}における結果の型を明記するためには、
クラスオブジェクトがsymbolにバインドされていないので、引用符なしで
{\tt cons, string, integer-vector, float-vector}などのクラス名symbolを使う。

\begin{refdesc}

\funcdesc{elt}{sequence pos}{
{\bf elt}は、{\em sequence}の中の{\em pos}番目の位置の値を得たり、({\bf setf}と
ともに)置いたりする最も一般的な関数である。
{\em sequence}は、リストまたは任意のオブジェクト、{\tt bit, char, integer, float}の
ベクトルである。
{\bf elt}は、多次元の行列に適用できない。}

\funcdesc{length}{sequence}{
{\em sequence}の長さを返す。
ベクトルにおいて、{\bf length}は一定の時間で終了する。
しかし、リスト型においては、長さに比例した時間がかかる。
{\bf length}が、もし環状リストに適用されたとき、決して終了しない。
代わりに{\bf list-length}を使用すること。
もし、{\em sequence}がfill-pointerを持つ行列ならば、
{\bf length}は行列全体のサイズを返すのではなくfill-pointerを返す。
このような行列のサイズを知りたい場合には、{\bf array-total-size}を
使用すること。}

\funcdesc{subseq}{sequence start \&optional end}{
{\em sequence}の{\em start}番目から({\em end}$-$1)番目までをそっくりコピーした
列を作る。
{\em end}は、デフォルト値として{\em sequence}の長さをとる。}

\funcdesc{copy-seq}{sequence}{
{\em sequence}のコピーした列を作る。
このコピーでは、{\em sequence}のトップレベルの参照のみがコピーされる。
入れこリストのコピーには{\bf copy-tree}を使い、
再帰参照を持つような列のコピーには
{\bf copy-object}を使うこと。}

\funcdesc{reverse}{sequence}{
{\em sequence}の順番を逆にし、{\em sequence}と同じ型の新しい列を
返す。}

\funcdesc{nreverse}{sequence}{
{\bf nreverse}は、{\bf reverse}の破壊(destructive)バージョンである。
{\bf reverse}はメモリを確保するが、{\bf nreverse}はしない。}

\funcdesc{concatenate}{result-type \&rest sequences}{
全ての{\em sequence}を連結させる。
それぞれの{\em sequence}は、なにかの列型である。
{\bf append}と違って、最後の一つまで含めた全ての列がコピーされる。
{\em result-type}は、{\tt cons,string,vector,float-vector}などの
クラスである。}

\funcdesc{coerce}{sequence result-type}{
{\em sequence}の型を変更する。
例えば、{\tt (coerce '(a b c) vector) = \#(a b c)}や
{\tt (coerce "ABC" cons) = (a b c)}である。
{\em result-type}型の新しい列が作られ、
{\em sequence}のそれぞれの要素はその列にコピーされる。
{\em result-type}は、{\tt vector, integer-vector, float-vector, bit-vector, string, cons}
またはそれらの1つを継承したユーザー定義クラス
のうちの1つである。
{\bf coerce}は、{\em sequence}の型が{\em result-type}と同一である場合、コピーをする。}

\funcdesc{map}{result-type function seq \&rest more-seqs}{
{\em function}は、{\em seq}と{\em more-seqs}のそれぞれのN番目($N=0,1,\cdots$)の要素
からなるリストに
対して適用され、その結果は{\em result-type}の型の列に蓄積される。例えば、{\tt (map float-vector \#'(lambda (x) (* x x)) (float-vector 1 2 3))}のように書く。}

\funcdesc{fill}{sequence item \&key (start 0) (end (length sequence))}{
{\em sequence}の{\em start}番目から({\em end}$-$1)番目まで、{\em item}で満たす。}

\funcdesc{replace}{dest source \&key start1 end1 start2 end2}{
{\em dest}列の中の{\em start1}から{\em end1}までの要素が、
{\em source}列の中の{\em start2}から{\em end2}までの要素に置き換えられる。
{\em start1}と{\em start2}のデフォルト値はゼロで、
{\em end1}と{\em end2}のデフォルト値はそれぞれの列の長さである。
もし片方の列がもう一方よりも長いならば、
endは短い列の長さに一致するように縮められる。}

\funcdesc{sort}{sequence compare \&optional key}{
{\em sequence}は、Unixのquick-sortサブルーチンを使って破壊的に(destructively)
にソートされる。
{\em key}は、キーワードパラメータでなく、比較用のパラメータである。
同じ要素を持った列のソートをするときは十分気をつけること。
例えば、{\tt (sort '(1 1) \#'>)}は失敗する。なぜなら、1と1の比較は
どちらからでも失敗するからである。
この問題を避けるために、比較として{\tt \#'$>=$}か{\tt \#'$<=$}のような関数を用いる。}

\funcdesc{merge}{result-type seq1 seq2 pred \&key (key \#'identity)}{
2つの列{\em seq1}と{\em seq2}は、{\em result-type}型の1つの列に
合併され、それらの要素は{\em pred}に記述された比較を満足する。}

\funcdesc{merge-list}{list1 list2 pred key}{
2つのリストを合併させる。{\bf merge}と違って、一般列は引数として
許可されないが、{\bf merge-list}は{\bf merge}より実行が速い}

\end{refdesc}

次の関数は、1つの基本関数と-ifや-if-notを後に付けた変形関数から成る。
基本形は、少なくともitemとsequenceの引数を持つ。
sequenceの中のそれぞれの要素とitemを比較し、
何かの処理をする。
例えば、インデックスを探したり、
現れる回数を数えたり、itemを削除したりなど。
変形関数は、predicateとsequenceの引数を持つ。
sequenceのそれぞれの要素にpredicateを適用し、
もしpredicateがnon-NILを返したとき(-if version)、
またはNILを返したとき(-if-not version)に何かをする。

\begin{refdesc}

\funcdesc{position}{item seq \&key start end test test-not key (count 1)}{
{\em seq}の中から{\em item}と同一な要素を探し、
その要素の中で{\em :count}番目に現れた要素の
インデックスを返す。
その探索は、{\em :start}番目の要素から始め、それ以前の要素は無視する。
デフォルトの探索は、{\tt eql}で実行されるが、
{\em test}か{\em test-not}パラメータで変更できる。}

\fundesc{position-if}{predicate seq \&key start end key}

\fundesc{position-if-not}{predicate seq \&key start end key}

\funcdesc{find}{item seq \&key start end test test-not key (count 1)}{
{\em seq}の中の{\em start}番目の要素から
{\em :end}番目の要素までの間で要素を探し、
その探された要素の内、{\em :count}番目の要素を返す。
その要素は、{\em :test}か{\em :test-not}に{\tt \#'eql}
以外のものが記述されていないなら、{\em item}と同じものである。}

\funcdesc{find-if}{predicate seq \&key start end key (count 1)}{
{\em seq}の要素の中で{\em predicate}がnon-NILを返す要素の内、
{\em :count}番目の要素を返す。}

\fundesc{find-if-not}{predicate seq \&key start end key}

\funcdesc{count}{item seq \&key start end test test-not key}{
{\em seq}の中の{\em :start}番目から{\em :end}番目までの要素に{\em item}が
何回現れるか数える。}

\funcdesc{count-if}{predicate seq \&key start end key}{
{\em predicate}がnon-NILを返す{\em seq}内の要素数を数える。}

\fundesc{count-if-not}{predicate seq \&key start end key}

\funcdesc{remove}{item seq \&key start end test test-not key count}{
{\em seq}の中の{\em :start}番目から{\em :end}番目までの要素のなかで、
{\em item}と同一の要素を探し、{\em :count}
(デフォルト値は∞)番目までの要素を削除した新しい列を作る。
もし、{\em item}が一回のみ現れることが確定しているなら、
無意味な探索を避けるために、{\em :count=1}を指定すること。}

\fundesc{remove-if}{predicate seq \&key start end key count}

\fundesc{remove-if-not}{predicate seq \&key start end key count}

\funcdesc{remove-duplicates}{seq \&key start end key test test-not count}{
{\em seq}の中から複数存在するitemを探し、その中の1つだけを残した新しい列を作る。}

\funcdesc{delete}{item seq \&key start end test test-not key count}{
{\em delete}は、{\em seq}自体を修正し、新しい列を作らないことを除いては、
{\bf remove}同じである。
もし、{\em item}が一回のみ現れることが確定しているなら、
無意味な探索を避けるために、{\em :count=1}を指定すること。}

\fundesc{delete-if}{predicate seq \&key start end key count}

\funcdesc{delete-if-not}{predicate seq \&key start end key count}{
{\bf remove}や{\bf delete}の{\em :count}デフォルト値は、1,000,000である。
もし列が長く、削除したい要素が一回しか現れないときは、
{\em :count}を1と記述すべきである。}

\funcdesc{substitute}{newitem olditem seq \&key start end test test-not key count}{
{\em seq}の中で{\em :count}番目に現れた{\em olditem}を{\em newitem}に置き換えた
新しい列を返す。
デフォルトでは、全ての{\em olditem}を置き換える。}

\begin{verbatim}
(substitute #\Space #\_ "Euslisp_euslisp") ;; => "Euslisp euslisp"
\end{verbatim}

\fundesc{substitute-if}{newitem predicate seq \&key start end key count}

\fundesc{substitute-if-not}{newitem predicate seq \&key start end key count}

\funcdesc{nsubstitute}{newitem olditem seq \&key start end test test-not key count}{
{\em seq}の中で{\em count}番目に現れた{\em olditem}を{\em newitem}に置き換え、
元の列{\em seq}に返す。デフォルトでは、全ての{\em olditem}を置き換える。}

\fundesc{nsubstitute-if}{newitem predicate seq \&key start end key count}

\fundesc{nsubstitute-if-not}{newitem predicate seq \&key start end key count}

\end{refdesc}

\newpage

\subsection{リスト}

\begin{refdesc}

\funcdesc{listp}{object}{
オブジェクトがconsのインスタンスかもしくはNILならば、Tを返す。}

\funcdesc{consp}{object}{
{\tt (not (atom object))}と同一である。{\tt (consp '())}はNILである。}

\funcdesc{car}{list}{
{\em list}の最初の要素を返す。NILの{\bf car}はNILである。
atomの{\bf car}はエラーとなる。{\tt (car '(1 2 3)) = 1}}

\funcdesc{cdr}{list}{
{\em list}の最初の要素を削除した残りのリストを返す。NILの{\bf cdr}はNILである。
atomの{\bf cdr}はエラーとなる。{\tt (cdr '(1 2 3)) = (2 3)}}

\funcdesc{cadr}{list}{
{\tt (cadr list) = (car (cdr list))}}

\funcdesc{cddr}{list}{
{\tt (cddr list) = (cdr (cdr list))}}

\funcdesc{cdar}{list}{
{\tt (cdar list) = (cdr (car list))}}

\funcdesc{caar}{list}{
{\tt (caar list) = (car (car list))}}

\funcdesc{caddr}{list}{
{\tt (caddr list) = (car (cdr (cdr list)))}}

\funcdesc{caadr}{list}{
{\tt (caadr list) = (car (car (cdr list)))}}

\funcdesc{cadar}{list}{
{\tt (cadar list) = (car (cdr (car list)))}}

\funcdesc{caaar}{list}{
{\tt (caaar list) = (car (car (car list)))}}

\funcdesc{cdadr}{list}{
{\tt (cdadr list) = (cdr (car (cdr list)))}}

\funcdesc{cdaar}{list}{
{\tt (cdaar list) = (cdr (car (car list)))}}

\funcdesc{cdddr}{list}{
{\tt (cdddr list) = (cdr (cdr (cdr list)))}}

\funcdesc{cddar}{list}{
{\tt (cddar list) = (cdr (cdr (car list)))}}

\funcdesc{first}{list}{
{\em list}の最初の要素を取り出す。
{\bf second, third, fourth, fifth, sixth, seventh, eighth}もまた定義されている。{\tt (first list) = (car list)}}

%\funcdesc{second}{list}{{\tt (second list) = (cadr list)}}
%\funcdesc{third}{list}{{\tt (third list) = (caddr list)}}
%\funcdesc{fourth}{list}{{\tt (fourth list) = (car (cdddr list))}}
%\funcdesc{fifth}{list}{{\tt (fifth list) = (cadr (cdddr list))}}

\funcdesc{nth}{count list}{
{\em list}内の{\em count}番目の要素を返す。
{\tt (nth 1 list)}は、{\tt (second list)}あるいは{\tt (elt list 1)}と等価である。}

\funcdesc{nthcdr}{count list}{
{\em list}に{\bf cdr}を{\em count}回適用した後のリストを返す。}

\funcdesc{last}{list}{
{\em list}の最後の要素でなく、最後のconsを返す。}

\funcdesc{butlast}{list \&optional (n 1)}{
{\em list}の最後から{\em n}個の要素を削除したリストを返す。}

\funcdesc{cons}{car cdr}{
{\tt car}が{\em car}で{\tt cdr}が{\em cdr}であるような新しいconsを作る。}

\funcdesc{list}{\&rest elements}{
makes a list of {\em elements}.}

\funcdesc{list*}{\&rest elements}{
{\em element}を要素とするリストを作る。しかし、最後の要素は{\bf cons}されるため、
atomであってはならない。
例えば、{\tt (list* 1 2 3 '(4 5)) = (1 2 3 4 5)}である。}

\funcdesc{list-length}{list}{
{\em list}の長さを返す。{\em list}は、環状リストでも良い。}

\funcdesc{make-list}{size \&key initial-element}{
{\em size}長さで要素が全て{\em :initial-element}のリストを作る。}

\funcdesc{rplaca}{cons a}{
{\em cons}の{\bf car}を{\em a}に置き換える。
{\bf setf}と{\bf car}の使用を推薦する。
{\tt (rplaca cons a) = (setf (car cons) a)}}

\funcdesc{rplacd}{cons d}{
{\em cons}の{\bf cdr}を{\em d}に置き換える。
{\bf setf}と{\bf cdr}の使用を推薦する。
{\tt (rplacd cons d) = (setf (cdr cons) d)}}

\funcdesc{memq}{item list}{
{\bf member}に似ている。しかしテストはいつも{\tt eq}で行う。}

\funcdesc{member}{item list \&key key (test \#'eq) test-not}{
{\em list}の中から条件にあった要素を探す。
{\em list}の中から{\em item}を探索し、{\em :test}の条件にあったものがなければNILを返す。
見つかったならば、それ以降をリストとして返す。この探索は、最上位のリストに対して
行なわれる。{\em :test}のデフォルトは{\tt \#'eq}である。
{\tt (member 'a '(g (a y) b a d g e a y))=(a d g e a y)}}

\fundesc{assq}{item alist}

\funcdesc{assoc}{item alist \&key key (test \#'eq) test-not}{
{\em alist}の要素の{\bf car}が{\em :test}の条件にあった最初のものを返す。
合わなければ、NILを返す。
{\em :test}のデフォルトは{\tt \#'eq}である。
{\tt (assoc '2 '((1 d t y)(2 g h t)(3 e x g))=(2 g h t)}}

\funcdesc{rassoc}{item alist}{
{\bf cdr}が{\em item}に等しい{\em alist}のなかの最初の組を返す。}

\funcdesc{pairlis}{l1 l2 \&optional alist}{
{\em l1}と{\em l2}の中の一致する要素を対にしたリストを作る。
もし{\em alist}が与えられたとき、
{\em l1}と{\em l2}から作られた対リストの最後に連結させる。}

\funcdesc{acons}{key val alist}{
{\em alist}に{\em key val}の組を付け加える。
{\tt (cons (cons key val) alist)}と同等である。}

\funcdesc{append}{\&rest list}{
新しいリストを形成するために{\em list}を連結させる。
最後のリストを除いて、{\em list}のなかの全ての要素はコピーされる。}

\funcdesc{nconc}{\&rest list}{
それぞれの{\em list}の最後の{\bf cdr}を置き換える事によって、{\em list}を
破壊的に(destructively)連結する。}

\funcdesc{subst}{new old tree}{
{\em tree}の中のすべての{\em old}を{\em new}に置き換える。}

\funcdesc{flatten}{complex-list}{
atomやいろんな深さのリストを含んだ{\em complex-list}を、
1つの線形リストに変換する。そのリストは、
{\em complex-list}の中のトップレベルに全ての要素を置く。
{\tt (flatten '(a (b (c d) e))) = (a b c d e)}}

\macrodesc{push}{item place}{
{\em place}にバインドされたスタック(リスト)に{\em item}を置く。}

\macrodesc{pop}{stack}{
{\em stack}から最初の要素を削除し、それを返す。
もし{\em stack}が空(NIL)ならば、NILを返す。}

\macrodesc{pushnew}{item place \&key test test-not key}{
もし{\em item}が{\em place}のメンバーでないなら、
{\em place}リストに{\em item}を置く。
{\em :test}, {\em :test-not}と{\em :key}引数は、
{\bf member}関数に送られる。}

\funcdesc{adjoin}{item list}{
もし{\em item}が{\em list}に含まれてないなら、{\em list}の最初に付け加える。}

\funcdesc{union}{list1 list2 \&key (test \#'eq) test-not (key \#'identity)}{
2つのリストの和集合を返す。}

\funcdesc{subsetp}{list1 list2 \&key (test \#'eq) test-not (key \#'identity)}{
{\em list1}が{\em list2}の部分集合であること、すなわち、
{\em list1}のそれぞれの要素が{\em list2}のメンバーであることをテストする。}

\funcdesc{intersection}{list1 list2 \&key (test \#'eq) test-not (key \#'identity)}{
2つのリスト{\em list1}と{\em list2}の積集合を返す。}

\funcdesc{set-difference}{list1 list2 \&key (test \#'eq) test-not (key \#'identity)}{
{\em list1}にのみ含まれていて
{\em list2}に含まれていない要素からなるリストを返す。}

\funcdesc{set-exclusive-or}{list1 list2 \&key (test \#'eq) test-not (key \#'identity)}{
{\em list1}および{\em list2}にのみ現れる要素からなるリストを返す。}

\funcdesc{list-insert}{item pos list}{
{\em list}の{\em pos}番目の要素として{\em item}を挿入する
(元のリストを変化させる)。
もし{\em pos}が{\em list}の長さより大きいなら、{\em item}は最後に
{\bf nconc}される。
{\tt (list-insert 'x 2 '(a b c d)) = (a b x c d)}}

\funcdesc{copy-tree}{tree}{
入れこリストである{\em tree}のコピーを返す。
しかし、環状参照はできない。環状リストは、
{\bf copy-object}でコピーできる。
実際に、{\bf copy-tree}は{\tt (subst t t tree)}と簡単に記述される。
なお{\bf copy-tree}は{\em list}しかコピーできず、
{\em integer-vector, float-vector}などは{\bf copy-seq}を使用しないとコピーできず、
{\em interger-vector, float-vector}などを含む{\em list}は{\bf copy-object}を
使用しないとディープコピーできない。}

\funcdesc{mapc}{func arg-list \&rest more-arg-lists}{
{\em arg-list}や{\em more-arg-lists}それぞれのN番目($N=0,1,\cdots$)の要素からなるリストに
{\em func}を適用する。
適用結果は無視され、{\em arg-list}が返される。}

\funcdesc{mapcar}{func \&rest arg-list}{
{\em arg-list}のそれぞれの要素に{\em func}を{\bf map}し、
その全ての結果のリストを作る。例えば、{\tt (mapcar \#'(lambda (x) (* x x)) '(1 2 3))}のように書く。
{\bf mapcar}を使う前に、{\bf dolist}を試すこと。}

\funcdesc{mapcan}{func arg-list \&rest more-arg-lists}{
{\em arg-list}のそれぞれの要素に{\em func}を{\bf map}し、
{\bf nconc}を用いてその全ての結果のリストを作る。
{\bf nconc}はNILに対して何もしないため、
{\bf mapcan}は、{\em arg-list}の要素にフィルタをかける(選択する)
のに合っている。}

\end{refdesc}
\newpage

\subsection{ベクトルと行列}

7次元以内の行列が許可されている。
1次元の行列は、ベクトルと呼ばれる。
ベクトルとリストは、列としてグループ化される。
もし、行列の要素がいろんな型であったとき、その行列は一般化されていると言う。
もし、行列がfill-pointerを持ってなく、他の行列で置き換えられなく、
拡張不可能であるなら、その行列は簡略化されたと言う。

全ての行列要素は、{\bf aref}により取り出すことができ、{\bf aref}を用いて{\bf setf}
により設定することができる。
しかし、一次元ベクトルのために簡単で高速なアクセス関数がある。
{\bf svref}は一次元一般ベクトル、{\bf char}と{\bf schar}は
一次元文字ベクトル(文字列)、{\bf bit}と{\bf sbit}は
一次元ビットベクトルのための高速関数である。
これらの関数はコンパイルされたとき、
アクセスはin-lineを拡張し、型チェックと境界チェックなしに実行される。

ベクトルもまたオブジェクトであるため、
別のベクトルクラスを派生させることができる。
5種類の内部ベクトルクラスがある。
{\tt vector, string, float-vector, integer-vector}と{\tt bit-vector}である。
ベクトルの作成を容易にするために、make-array関数がある。
要素の型は、{\tt :integer, :bit, :character, :float, :foreign}
かあるいはユーザーが定義したベクトルクラスの内の一つでなければならない。
{\em :initial-element}と{\em :initial-contents}のキーワード引数は、
行列の初期値を設定するために役に立つ。

\begin{refdesc}

\constdesc{array-rank-limit}{
7。行列の最大次元を示す。}

\constdesc{array-dimension-limit}{
\#x1fffffff。各次元の最大要素数を示す。
論理的な数であって、システムの物理メモリあるいは仮想メモリの大きさによって
制限される。}

\funcdesc{vectorp}{object}{
行列は1次元であってもベクトルではない。
{\em object}が{\tt vector, integer-vector, float-vector, string, bit-vector}
あるいはユーザーで定義したベクトルならTを返す。}

\funcdesc{vector}{\&rest elements}{
{\em elements}からなる一次元ベクトルを作る。}

\longdescription{関数}{make-array}{dims \&key \= (element-type vector) \\
\> initial-contents \\
\> initial-element \\
\> fill-pointer  \\
\> displaced-to \\
\> displaced-index-offset 0) \\
\> adjustable}{
ベクトルか行列を作る。
{\em dims}は、整数かリストである。
もし{\em dims}が整数なら、一次元ベクトルが作られる。}

\funcdesc{svref}{vector pos}{
{\em vector}の{\em pos}番目の要素を返す。
{\em vector}は、一次元一般ベクトルでなければならない。}

\funcdesc{aref}{vector \&rest indices}{
{\em vector}の{\em indices}によってインデックスされる要素を返す。
{\em indices}は、整数であり、{\em vector}の次元の数だけ指定する。
{\bf aref}は、非効率的である。なぜなら、{\em vector}の型に従うように
変更する必要があるためである。コンパイルコードの速度を改善する
ため、できるだけ型の宣言を与えるべきである。}

\funcdesc{vector-push}{val array}{
{\em array}のfill-pointer番目のスロットに{\em val}を保管する。
{\em array}は、fill-pointerを持っていなければならない。
{\em val}が保管された後、
fill-pointerは、次の位置にポイントを1つ進められる。
もし、行列の境界よりも大きくなったとき、エラーが報告される。}

\funcdesc{vector-push-extend}{val array}{
{\em array}のfill-pointerが最後に到達したとき、自動的に{\em array}のサイズが
拡張されることを除いては、{\bf vector-push}と同じである。}

\funcdesc{arrayp}{obj}{
もし{\em obj}が行列またはベクトルのインスタンスであるならTを返す。}

\funcdesc{array-total-size}{array}{
{\em array}の要素数の合計を返す。}

\funcdesc{fill-pointer}{array}{
{\em array}のfill-pointerを返す。
fill-pointerを持っていなければNILを返す。}

\funcdesc{array-rank}{array}{
{\em array}の次元数を返す。}

\funcdesc{array-dimensions}{array}{
{\em array}の各次元の要素数をリストで返す。}

\funcdesc{array-dimension}{array axis}{
{\bf array-dimension}は、{\em array}の{\em axis}
番目の次元を返す。{\em axis}はゼロから始まる。}

\funcdesc{bit}{bitvec index}{
{\em bitvec}の{\em index}番目の要素を返す。
ビットベクトルの要素を変更するには、{\bf setf}と{\bf bit}を使用すること。}

\fundesc{bit-and}{bits1 bits2 \&optional result}

\fundesc{bit-ior}{bits1 bits2 \&optional result}

\fundesc{bit-xor}{bits1 bits2 \&optional result}

\fundesc{bit-eqv}{bits1 bits2 \&optional result}

\fundesc{bit-nand}{bits1 bits2 \&optional result}

\fundesc{bit-nor}{bits1 bits2 \&optional result}

\funcdesc{bit-not}{bits1 \&optional result}{
同じ長さの{\em bits1}と{\em bits2}というビットベクトルにおいて、
それらのand, inclusive-or,
exclusive-or, 等価, not-and, not-orとnotがそれぞれ返される。}

\end{refdesc}

\newpage

\subsection{文字と文字列}

EusLispには、文字型がない。
文字は、{\tt integer}によって表現されている。
ファイル名を現わす文字列を扱うためには、
\ref{Pathnames}節に書かれている{\bf pathname}を使うこと。

\begin{refdesc}

\funcdesc{digit-char-p}{ch}{
もし{\em ch}が{\tt \#$\backslash$0}〜{\tt \#$\backslash$9}ならTを返す。}

\funcdesc{alpha-char-p}{ch}{
もし{\em ch}が{\tt \#$\backslash$A}〜{\tt \#$\backslash$Z}または
{\tt \#$\backslash$a}〜{\tt \#$\backslash$z}なら、Tを返す。}

\funcdesc{upper-case-p}{ch}{
もし{\em ch}が{\tt \#$\backslash$A}〜{\tt \#$\backslash$Z}なら、Tを返す。}

\funcdesc{lower-case-p}{ch}{
もし{\em ch}が{\tt \#$\backslash$a}〜{\tt \#$\backslash$z}なら、Tを返す。}

\funcdesc{alphanumericp}{ch}{
もし{\em ch}が{\tt \#$\backslash$0}〜{\tt \#$\backslash$9}、
{\tt \#$\backslash$A}〜{\tt \#$\backslash$Z}または
{\tt \#$\backslash$a}〜{\tt \#$\backslash$z}なら、Tを返す。}

\funcdesc{char-upcase}{ch}{
{\em ch}を大文字に変換する。}

\funcdesc{char-downcase}{ch}{
{\em ch}を小文字に変換する。}

\funcdesc{char}{string index}{
{\em string}の{\em index}番目の文字を返す。}

\funcdesc{schar}{string index}{
{\em string}から文字を抜き出す。
{\em string}の型が明確に解っていて、型チェックを要しないときのみ、{\bf schar}
を使うこと。}

\funcdesc{stringp}{object}{
{\em object}がバイト(256より小さい正の整数)のベクトルなら、Tを返す。}

\funcdesc{string-upcase}{str \&key start end}{
{\em str}を大文字の文字列に変換して、新しい文字列を返す。}

\funcdesc{string-downcase}{str \&key start end}{
{\em str}を小文字の文字列に変換して、新しい文字列を返す。}

\funcdesc{nstring-upcase}{str}{
{\em str}を大文字の文字列に変換し、元に置き換える。}

\funcdesc{nstring-downcase}{str \&key start end}{
{\em str}を小文字の文字列に変換し、元に置き換える。}

\funcdesc{string=}{str1 str2 \&key start1 end1 start2 end2}{
もし{\em str1}が{\em str2}と等しいとき、Tを返す。
{\bf string=}は、大文字・小文字を判別する。}

\funcdesc{string-equal}{str1 str2 \&key start1 end1 start2 end2}{
{\em str1}と{\em str2}の等価性をテストする。
{\bf string-equal}は、大文字・小文字を判別しない。}

\funcdesc{string}{object}{
{\em object}の文字列表現を得る。
もし{\em object}が文字列なら、{\em object}が返される。
もし{\em object}がsymbolなら、その{\tt pname}がコピーされ、返される。
{\tt (equal (string 'a) (symbol-pname 'a))==T}であるが、
{\tt (eq (string 'a) (symbol-pname 'a))==NIL}である。
もし{\em object}が数値なら、それを文字列にしたものが返される
(これはCommon Lispと非互換である)。
もっと複雑なオブジェクトから文字列表現を得るためには、
最初の引数をNILにした{\bf format}関数を用いること。}

\fundesc{string$<$}{str1 str2}

\fundesc{string$<=$}{str1 str2}

\fundesc{string$>$}{str1 str2}

\fundesc{string$>=$}{str1 str2}

\fundesc{string-left-trim}{bag str}

\funcdesc{string-right-trim}{bag str}{
{\em str}は、左(右)から探索され、もし{\em bag}リスト内の文字を含んでいるなら、
その要素を削除する。
一旦{\em bag}に含まれない文字が見つかると、その後の探索は中止され、
{\em str}の残りが返される。}

\funcdesc{string-trim}{bag str}{
{\em bag}は、文字コードの列である。
両端に{\em bag}に書かれた文字を含まない{\em str}のコピーが作られ、返される。}

\funcdesc{substringp}{sub string}{
{\em sub}文字列が{\em string}に部分文字列として含まれるなら、Tを返す。
大文字・小文字を判別しない。}

\end{refdesc}

\subsubsection{日本語の扱い方}
{\bf euslisp}で日本語を扱いたい時、文字コードは{\bf UTF-8}である必要がある。

例えば{\bf concatenate}を用いると、リストの中の日本語を連結することが出来る。
{\bf ROS}のトピックとして一つずつ取ってきた日本語を、連結することで一つの{\bf string}型の言葉に変換したい時などに便利である。

(concatenate string "け" "ん" "し" "ろ" "う") 
→ "けんしろう"

最初から全ての文字がリストに入っていて、文字を連結したい時はこのようにすればよい。

(reduce {\#}'(lambda (val1 val2) (concatenate string val1 val2)) (list "我" "輩" "は" "ピ" "ー" "ア" "ー" "ル" "ツ" "ー" "で" "あ" "る"))

→ "我輩はピーアールツーである"

{\bf coerce}を用いて、次のように書くことも出来る。

(coerce (append (coerce "私はナオより" cons) (coerce "背が高い" cons)) string)

→ "私はナオより背が高い"

\subsection{Foreign String}
{\bf foreign-string}は、EusLispのヒープ外にあるバイトベクトルの1種である。
普通の文字列は、長さとバイトの列を持ったオブジェクトであるが、
foreign-stringは、長さと文字列本体のアドレスを持っている。
foreign-stringは文字列であるが、
いくつかの文字列および列に対する関数は適用できない。
{\bf length}、{\bf aref}、{\bf replace}、{\bf subseq}と{\bf copy-seq}だけが
foreign-stringを認識し、 
その他の関数の適用はクラッシュの原因となる恐れがある。

foreign-stringは、/dev/a??d??(??は32あるいは16)の特殊ファイルで与えられる
I/O空間を参照することがある。
そのデバイスがバイトアクセスにのみ応答するI/O空間の一つに
割り当てられた場合、
{\bf replace}は、いつもバイト毎に要素をコピーする。
メモリのlarge chunkを連続的にアクセスしたとき、比較的に遅く動作する。。

\begin{refdesc}

\funcdesc{make-foreign-string}{address length}{
{\em address}の位置から{\em length}バイトの
{\bf foreign-string}のインスタンスを作る。
例えば、{\tt (make-foreign-string (unix:malloc 32) 32)}は、
EusLispのヒープ外に位置する32バイトメモリを参照部分として作る。}

\end{refdesc}
\newpage

\subsection{ハッシュテーブル}

hash-tableは、キーで連想される値を探すためのクラスである({\bf assoc}でもできる)。
比較的大きな問題において、hash-tableはassocより良い性能を出す。
キーと値の組数が増加しても探索に要する時間は、一定のままである。
簡単に言うと、hash-tableは100以上の要素から探す場合に用い、
それより小さい場合はassocを用いるべきである。

hash-tableは、テーブルの要素数がrehash-sizeを越えたなら、自動的に拡張される。
デフォルトとして、テーブルの半分が満たされたとき拡張が起こるようになっている。
{\bf sxhash}関数は、オブジェクトのメモリアドレスと無関係なハッシュ値を
返し、オブジェクトが等しい(equal)ときのハッシュ値はいつも同じである。
それで、hash-tableはデフォルトのハッシュ関数に{\bf sxhash}を使用している
ので、再ロード可能である。
{\bf sxhash}がロバストで安全な間は、
列やtreeの中のすべての要素を探索するため、比較的に遅い。
高速なハッシュのためには、アプリケーションにより他の特定のハッシュ関数を
選んだ方がよい。
ハッシュ関数を変えるためには、hash-tableに{\tt :hash-function}メッセージを
送信すれば良い。
簡単な場合、ハッシュ関数を{\tt \#'sxhash}から
{\tt \#'sys:address}に変更すればよい。
EusLisp内のオブジェクトのアドレスは
決して変更されないため、{\tt \#'sys:address}を設定することができる。

\begin{refdesc}

\funcdesc{sxhash}{obj}{
{\em obj}のハッシュ値を計算する。
{\bf equal}な2つのオブジェクトでは、同じハッシュ値を生じること
が保証されている。
symbolなら、そのpnameに対するハッシュ値を返す。
numberなら、その{\tt integer}表現を返す。
listなら、その要素全てのハッシュ値の合計が返される。
stringなら、それぞれの文字コードの合計をシフトしたものが返される。
その他どんなオブジェクトでも、{\bf sxhash}はそれぞれのスロットのハッシュ値を
再帰的呼出しで計算し、それらの合計を返す。}

\funcdesc{make-hash-table}{\&key (size 30) (test \#'eq) (rehash-size 2.0)}{
hash-tableを作り、返す。}


\funcdesc{gethash}{key htab}{
{\em htab}の中から{\em key}と一致する値を得る。
{\bf gethash}は、{\bf setf}を組み合せることにより{\em key}に値を設定する
ことにも使用される。
hash-tableに新しい値が登録され、そのテーブルの埋まったスロットの数が
1/rehash-sizeを越えたとき、hash-tableは自動的に2倍の大きさに拡張される。}

\funcdesc{remhash}{key htab}{
{\em htab}の中から{\em key}で指定されたハッシュ登録を削除する。}

\funcdesc{maphash}{function htab}{
{\em htab}の要素全てを{\em function}で{\bf map}する。}

\funcdesc{hash-table-p}{x}{
もし{\em x}がhash-tableクラスのインスタンスなら、Tを返す。}

\classdesc{hash-table}{object}{(key value count \\
\> hash-function test-function \\
\> rehash-size empty deleted)}{
hash-tableを定義する。
{\em key}と{\em value}は大きさが等しい一次元ベクトルである。
{\em count}は、{\em key}や{\em value}が埋まっている数である。
{\em hash-function}のデフォルトは{\bf sxhash}である。
{\em test-function}のデフォルトは{\bf eq}である。
{\em empty}と{\em deleted}は、{\em key}ベクトルのなかで空または削除された
数を示すsymbol(パッケージに収容されていない)である。}

\methoddesc{:hash-function}{newhash}{
このhash-tableのハッシュ関数を{\em newhash}に変更する。
{\em newhash}は、1つの引数を持ち、{\tt integer}を返す関数でなければならない。
{\em newhash}の1つの候補として{\bf sys:address}がある。}

\end{refdesc}

\newpage
\subsection{キュー}

\begin{refdesc}

queue はFIFO(first-in first-out)形式で要素の追加や検索を可能にするデー
タ構造である。
carがqueueの実体であり、このqueueのインスタンスをqとすると,その配列
は(print (car q))で表示できる。cdrに最も最近に追加された要素が確保される。

\classdesc{queue}{cons}{(car cdr)}{
FIFO queueを定義する。}


\methoddesc{:init}{}{
queueを初期化する。}

\methoddesc{:enqueue}{val}{
このqueueにxを最新の要素として追加する。}

\methoddesc{:dequeue}{\&optional (error-p nil)}{
このqueueの最も過去に追加された要素を返し、
queueから削除する。queueが空の場合,error-pが
NILでない場合はエラーを表示し、それ以外の場合はNILを返す。}

\methoddesc{:empty?}{}{
このqueueが空の場合、Tを返す。}

\methoddesc{:length}{}{
このqueueの長さを返す。}

\methoddesc{:trim}{s}{
このqueueの要素のうち古いものを破棄し、queueの長さをsとする。}

\methoddesc{:search}{item \&optional (test \#'equal)}{
このqueueにitemが存在するか調べる。もし無ければNILを返す。}

\methoddesc{:delete}{item \&optional (test \#'equal) (count 1)}{
このqueueからitemと同一の要素を探し、:count番目までの要素を削除する。}

\methoddesc{:first}{}{
このqueueの先頭の要素(最も過去に追加された要素)を返す。}

\methoddesc{:last}{}{
このqueueの最後の要素(最も最近に追加された要素)を返す。}

\end{refdesc}
\newpage
