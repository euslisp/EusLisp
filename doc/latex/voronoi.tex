\subsection{Voronoi Diagram of Polygons}

\begin{refdesc}

\hfill {\em Author: Philippe PIGNON, ETL Guest Researcher}

The program is written in COMMON LISP. I used the method of Fortune, 
 "A sweepline algorithm for Voronoi diagrams", in Proceedings of
the 2nd Annual ACM symposium on computational geometry, 1986, 313-322.
I adapted it to the polygonal case. This is a sample file with short
explanations
This program was written under Electrotechnical EUSLISP environment,
so graphic connections are provided for it. 
However, you can use it with any COMMON-LISP; you'll then have to write your
own display functions to replace those given in utilities.l file (see below)

{\bf PURPOSE:} Computation of the voronoi diagram of a set of polygons.
Please read the above quoted reference to understand the vocabulary and
method used. No explanations about the program itself  will be given here.

{\bf INPUT:} A list of polygons coordinates plus an enclosing frame.
\begin{verbatim}
DATA= (
       (x11 y11 x12 y12 x13 y13 ...) first polygon,
                                     counterclocwise enumeration of vertices
       (x21 y21 x22 y22 x23 y23 ...) second polygon
               ... 
       (xn1 yn1 xn2 yn2 xn3 yn3 ...) nth polygon
	     
       (xf1 yf1 xf2 yf2 xf3 yf3 xf4 yf4) enclosing frame
      )
\end{verbatim}
Enclosing frame can occur anywhere in data, and should be clockwise enumerated
for outside-inside marking consistency (see below). Polygons must be simple,
non intersecting. Aligned or flat edges are not accepted. 
Neither are isolated points or segments.

{\bf OUTPUT:} *diagram*: a list of doubly connected edges list
(cf utilities.l file). Each edge is a symbol, with  property list
including the following fields:
\begin{verbatim}
(start <pointer to a vertex>)
       (end <pointer to a vertex>)
       (pred <pointer to an edge>)
       (succ <pointer to an edge>)
       (left <pointer to a site>)
       (right <pointer to a site>)
       (type <:endpoint or :point-point or :segment-segment or :point-segment>)
       (outflag <t or nil>)
\end{verbatim}
A $vertex$ is a symbol whose property list contains the field "pos".
This field itself contains a cons $(x y)$, (real) planar coordinates
of the vertex.
$Pred$ and $succ$ field give counterclockwise predecessor and successor
according to the dcel formalism (see Shamos and Preparata,
Computational Geometry: An introduction, 1985, pp 15-17).
A $site$ is also a symbol, whose property list also contains relevant
information.
Sites describe original input data; they can be of type
:point (a polygon vertex) or segment (a polygon edge).

$Type$ is the gender of the bisector, determined by the type of the sites
it separates.
By convention, outside is the right side of a start-end edge.
The voronoi diagram computes ouside as well as inside bisectors.
Sort on outflag to keep the ones you want.

\funcdesc{pv}{data}{
Compute the Voronoi diagram of polygons from the $data$ with the above format.}

{\bf SAMPLE:}
In order to run the program on a short  sample, 
please perform the following steps:
\\
0- Copy the following files in your environment:
\\
\begin{tabular}{ll}
utilities.l & Geometric utility functions, plus EUSX graphic functions\\
polygonalvoronoi.l & The program.\\
testdata.l & Demonstration data, with the above format.  
\end{tabular}
\\
1- If you do not use EUS, edit the utilities.l file and
modify the "compatibility package" according to the instructions.
\\
2- Compile and/or load the following 3 files:
\\
\begin{tabular}{ll}
utilities.l\\
polygonalvoronoi.l\\
testdata.l & This file contains demonstration data,with the above format
\end{tabular}
\\
3- (pv demoworld)  run the program on demonstration data. 
The global variable *diagram* contains the bisectors of the voronoi diagram.

Under EUSX only (eus with XWindow interface), do the following
to display the resulting diagram:
\begin{verbatim}
       (make-display)          ;;Initializes the *display* window object
       (dps demoworld *thick*) ;; Shows original data in thick lines
       (dbs *diagram*)         ;; Shows the result
\end{verbatim}

\end{refdesc}

\newpage